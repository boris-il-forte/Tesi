% Adaptation of the POLIMI beamer template based on Politecnico di
% Milano - Department of Electronics and Computer Science - 
% powerpoint template. Original NTNU beamer template by H�vard Berland - http://www.pvv.ntnu.no/~berland/ntnubeamer/ 
%
% author: Luigi Malag� <luigi.malago@gmail.com> 
% 01-01-2014
%


\documentclass[mathserif,t,screen]{beamer}
\usepackage[T1]{fontenc}
\usepackage[latin1]{inputenc}

\usepackage{enumerate}
\usepackage{graphicx}
\usepackage{amsmath}
\usepackage{amssymb}
\usepackage{amsfonts}
\usepackage{mathrsfs}
\usepackage{MnSymbol}
\usepackage{multirow} 
\usepackage{tabls}

\usepackage{macros}

\usepackage{tweaklist}
\renewcommand{\itemhook}{\setlength{\topsep}{0pt}%
  \setlength{\itemsep}{1pt}}

\usepackage{kbordermatrix}
\setlength{\kbrowsep}{0pt}

\newcommand{\p}{\phantom{-}1}
\newcommand{\m}{-1}

\newcommand{\IconChecked}{\includegraphics[width=1.15cm]{gfx/checklist_cut}}
\newlength{\CheckedLength}
\setlength{\CheckedLength}{\labelwidth}\addtolength{\CheckedLength}{\labelsep}
\newcommand{\done}{\hspace*{-.365cm}\hspace*{\CheckedLength}\makebox[0cm][r]{\IconChecked}%
 \hspace*{-.125cm}}

\newcommand{\startproof}{\textit{Proof. }}

\newcommand{\norm}[1]{\left|\left|#1\right|\right|}

\newcommand{\eqdef}{\overset{\textup{def}}{=}}

\usetheme{polimienglish}
 
\title[Cognitive SLAM]{Cognitive SLAM:}


\subtitle{Knowledge-Based Simultaneous Localization and Mapping}

% location and date of the talk. The part in [] will be displayed on
% the bottom of every slide, the part between {} on the first page
\author[Davide Tateo]{\textbf{Davide Tateo}\\
Relatore: \textbf{Andrea Bonarini}
}


% \institute[POLIMI]{\textbf{DEPARTMENT OF ELECTRONICS AND INFORMATION}\\
% \url{http://www.dei.polimi.it}
% \vskip0.1cm
% \textbf{AIRLab} Artificial Intelligence and Robotics Laboratory\\
% \url{http://www.airlab.elet.polimi.it}}

% location and date of the talk. The part in [] will be displayed on
% the bottom of every slide, the part between {} on the first page
\date[03/10/2014]{3 Ottobre 2014}

\begin{document}

\polimititlepage
\addtocounter{framenumber}{-1}

%%%%%%%%%%%%%%%%%%%%%%%%%%%%%%%%%%%%%%%%%%%%%%%%%%%%%%%%%%%%%%%%%%%%%%%%%%%%%%%%%%%%%%%%%%%%%%%%%%%%%%%%%%%%%%%%%%%%

\begin{frame}
\tableofcontents
\end{frame}

%%%%%%%%%%%%%%%%%%%%%%%%%%%%%%%%%%%%%%%%%%%%%%%%%%%%%%%%%%%%%%%%%%%%%%%%%%%%%%%%%%%%%%%%%%%%%%%%%%%%%%%%%%%%%%%%%%%

\begin{frame}
\frametitle{Lists and spacing}

\begin{itemize}
\addtolength{\itemsep}{.2cm}

\item A

\item B

  \begin{itemize}
    \addtolength{\itemsep}{.1cm}

  \item[1.] one
\pause
  \item[2.] two

  \end{itemize}

\item C

\end{itemize}

\end{frame}

%%%%%%%%%%%%%%%%%%%%%%%%%%%%%%%%%%%%%%%%%%%%%%%%%%%%%%%%%%%%%%%%%%%%%%%%%%%%%%%%%%%%%%%%%%%%%%%%%%%%%%%%%%%%%%%%%%%

\begin{frame}
\frametitle{Columns and onslide}


\begin{columns}[t]
\column{3cm}
Column 1 

\onslide<3>{
\bigskip
\bigskip
 Appears with third column}
\pause{}
\column{3cm}
Column 2 
\pause
\column{3cm}
Column 3 
\end{columns}

\end{frame}


%%%%%%%%%%%%%%%%%%%%%%%%%%%%%%%%%%%%%%%%%%%%%%%%%%%%%%%%%%%%%%%%%%%%%%%%%%%%%%%%%%%%%%%%%%%%%%%%%%%%%%%%%%%%%%%%%%%%

\begin{frame}
\frametitle{Include eps graphics}

\bigskip
\centering{\includegraphics[height=2cm]{gfx/polimi-logo}}

\end{frame}


%%%%%%%%%%%%%%%%%%%%%%%%%%%%%%%%%%%%%%%%%%%%%%%%%%%%%%%%%%%%%%%%%%%%%%%%%%%%%%%%%%%%%%%%%%%%%%%%%%%%%%%%%%%%%%%%%%

\begin{frame}
\frametitle{Example of Proposition}


\begin{prop}[Name of Proposition]
This is a proposition
\begin{itemize}
\item[(i)] observation 1
\item[(ii)]observation 2
\end{itemize}
\end{prop}

\bigskip
\bigskip
Alternative style
\bigskip
\medskip

\begin{beamerboxesrounded}[shadow=true]{Name of Proposition}
This is another preposition
\end{beamerboxesrounded}

\end{frame}

%%%%%%%%%%%%%%%%%%%%%%%%%%%%%%%%%%%%%%%%%%%%%%%%%%%%%%%%%%%%%%%%%%%%%%%%%%%%%%%%%%%%%%%%%%%%%%%%%%%%%%%%%%%%%%%%%%%

\begin{frame}
\frametitle{Example of Theorem}

\begin{thm}[Th Name]
\label{thlabel} This is a theorem
\end{thm}
\bigskip
\startproof
\begin{displaymath}
  a = b = c
\end{displaymath}
\qed


\end{frame}

%%%%%%%%%%%%%%%%%%%%%%%%%%%%%%%%%%%%%%%%%%%%%%%%%%%%%%%%%%%%%%%%%%%%%%%%%%%%%%%%%%%%%%%%%%%%%%%%%%%%%%%%%%%%%%%%%%%

\begin{frame}
\frametitle{Some icons}

\begin{itemize}
\addtolength{\itemsep}{.2cm}
\item \done done
\item \hspace{.18cm} todo
\end{itemize}


\end{frame}

\end{document}
