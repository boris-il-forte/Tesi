\chapter{Il manuale utente}
\label{app:manuale}
\thispagestyle{empty}

\begin{quotation}
{\footnotesize
\noindent{\emph{Marty: Come puoi vedere il fulmine ha fatto saltare il microchip di controllo dei tempo-circuiti. L'allegato diagramma sch .. sch.. \\
Doc: Schematizzato. \\
Marty: ..schematizzato, ti permetterà di costruire un'unità sostitutiva con le componenti del 1955, riportando così la macchina del tempo a un perfetto funzionamento.
}
}
\begin{flushright}
Ritorno al Futuro, parte II
\end{flushright}
}
\end{quotation}
\vspace{0.5cm}

\section{Installazione e compilazione}

Per compilare il sistema in maniera completa è necessaria una istallazione completa di ROS. Si veda la guida ufficiale di ROS per conoscere i dettagli dell'istallazione del sistema. \`E necessario utilizzare ROS Hydro Medusa o superiore.  %%TODO link?
Inoltre è necessario installare i generatori di parser \verb|Flex| e \verb|Bison|.
\`E possibile compilare la libreria che implementa l'algoritmo di reasoning anche al di fuori di ROS, utilizzando semplicemente \verb|CMake|. In tal caso si raccomanda di installare anche le librerie \verb|Boost|.

Una volta installate le dipendenze necessarie, è possibile compilare i sorgenti. Per farlo, basta creare un workspace di catkin, aggiungere i package nella cartella src, e lanciare catkin\_make. Le compilazione avverrà nell'ordine corretto, rispettando le dipendenze tra i package.
\`E necessario aggiungere il file setup.bash nella cartella \verb|devel/| del workspace come script da eseguire nel vostro file .bashrc; ad esempio, se il workspace  si chiama catkin\_ws, e si trova nella home, basterà aggiungere al file .bashrc la linea:
\begin{verbatim}
source ~/catkin_ws/devel/setup.bash
\end{verbatim}

Per lanciare i singoli nodi si può utilizzare il programma rosrun. Ricordarsi di aggiungere gli argomenti da linea di comando e i parametri privati necessari subito dopo il nome del package e del nodo che si vuole lanciare. 
I parametri privati vengono specificati aggiungendo un ``\_'' prima del nome del parametro; il cui valore viene assegnato con il token ``:=''.

Ad esempio per lanciare il nodo c\_fuzzy\_reasoner nel package c\_fuzzy si può utilizzare il comando:

\begin{verbatim}
rosrun c_fuzzy c_fuzzy_reasoner -c knowledgebase.kb classifier.fuzzy
\end{verbatim}

Una alternativa per lanciare tutti i nodi contemporaneamente è usare un launchfile. Launchfile di esempio sono presenti nel package c\_slam.

Per lanciare l'intero sistema si può utilizzare il comando:

\begin{verbatim}
roslaunch c_slam c_slam.launch
\end{verbatim}


\section{Parametri e argomenti}

Di seguito sono elencati e spiegati i parametri utilizzati dai nodi. si veda il \autoref{cap:concetti} per informazioni riguardo agli algoritmi.

\subsection{c\_vision\_detector e c\_vision\_slam}

Questi nodi hanno bisogno necessariamente di parametri privati, senza i quali il loro funzionamento non è garantito.

\subsubsection{Canny}

Questi parametri sono utilizzati dall'algoritmo Canny Edge detector.

\begin{description}
 \item [canny/alpha] rappresenta il valore della soglia bassa, in proporzione alla soglia alta.
 \item [canny/apertureSize] rappresenta la dimensione del kernel con il quale applicare l'operatore di Sobel. Deve essere necessariamente un intero dispari e maggiore o uguale a 3. 
\end{description}

\subsubsection{Hough}

Questi parametri sono utilizzati per l'algoritmo di riconoscimento delle linee, detto trasformata di Hough probabilistica.

\begin{description}
 \item [hough/rho] risoluzione della distanza dall'origine delle rette espressa in pixel.
 \item [hough/teta] risoluzione dell'inclinazione delle rette, espressa in gradi.
 \item [hough/threshold] threshold per filtrare le linee dal rumore.
 \item [hough/minLineLenght] minima lunghezza delle linee riconosciute, espressa in pixel.
 \item [hough/maxLineGap] massima distanza tra due punti appartenenti alla stessa linea, in pixel.
\end{description}

\subsubsection{Line filtering}

Questi parametri vengono utilizzati dall'algoritmo che distingue le linee orizzontali e verticali dal rumore.

\begin{description}
 \item [filter/maxDeltaHorizontal] massima inclinazione delle rette orizzontali rispetto alla linea dell'orizzonte, in gradi.
 \item [filter/maxDeltaVertical] massima inclinazione delle rette verticali rispetto alla linea perpendicolare all'orizzonte, in gradi.
\end{description}


\subsubsection{Clustering}

Questi parametri servono per tarare il riconoscimento dei cluster. Sono utilizzati dagli algoritmi FAST, per riconoscere i keypoints, e DBSCAN per estrarre da essi i cluster.

\begin{description}
 \item [cluster/threshold] threshold per l'estrazione di keypoints dall'immagine.
 \item [cluster/minPoints] numero minimo di punti necessari a definire un cluster.
 \item [cluster/maxDistance] massima distanza tra i punti di un cluster.
\end{description}

Questi parametri non sono necessari nel nodo c\_vision\_detector.

\subsubsection{Classifier}

Questi parametri sono utilizzati dai nodi di visione quando vengono preparate le richieste di reasoning, e quindi sono di fatto utilizzati nel classificatore fuzzy.

\begin{description}
 \item [classifier/threshold] threshold da utilizzare nella classificazione delle feature.
\end{description}

\subsection{c\_fuzzy\_reasoner}

Per lanciare il reasoner è necessario specificare dei parametri da linea di comando.

\begin{description}
 \item [-h] stampa il messaggio di aiuto 
 \item [-r knowledgebase] crea il servizio di reasoning basato sulla knowledgebase specificata.  
 \item [-c knowledgebase classifier] crea il servizio di classificazione a partire dalla knowledgebase e dal classificatore specificati. 

\end{description}

