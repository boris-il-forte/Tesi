\chapter{Esempio di impiego}
\label{app:esempio}
\thispagestyle{empty}

\begin{quotation}
{\footnotesize
\noindent{\emph{Doc: Scusa, la rozzezza di questo modello ma non ho avuto il tempo di farlo in scala e di dipingerlo. \\
Marty: Va bene, va bene.
} }
\begin{flushright}
Ritorno al Futuro, parte III
\end{flushright}
}
\end{quotation}
\vspace{0.5cm}

Mostriamo in questo capitolo il sistema hardware e software utilizzato per testare i concetti, gli algoritmi e le metodologie presentate in questa tesi.

\section{Robot}
Il robot preso in considerazione per i nostri test sperimentali è un quadrotor. Abbiamo utilizzato l'A.R. Drone, prodotto da Parrot. 
Il quadrotor utilizzato è dotato di una telecamera HD con risoluzione $640\times 480$, tuttavia nella tesi sono state utilizzate immagini a $320\times 240$, a 30 frame al secondo.
La qualità delle immagini è scadente, e sono affette da pesanti problemi di refresh.
Il robot possiede anche una unità di misura inerziale economica e un magnetometro.
Inoltre il robot possiede una telecamera inferiore, di risoluzione $320\times 240$, la cui qualità dell'immagine è molto peggiore rispetto alla telecamera frontale, che tuttavia non è stata sfruttata per lo svolgimento di questa tesi.

\'E stato scelto di utilizzare questo quadrotor per alcuni validi motivi:
\begin{itemize}
 \item L'hardware è a basso costo, e il quadrotor si trova in commercio a un prezzo relativamente basso.
 \item Il robot in questione ha 6 gradi di libertà nello spazio, questo rende possibile lo studio di un sistema di localizzazzione e mapping in 3 dimensioni.
 \item Il robot è già integrato con ROS.
\end{itemize}


\section{Hardware e software di controllo}
Per controllare il quadrotor e lanciare l'algoritmo di controllo è stato utilizzato un portatile dotato di processore Intel Core i7-4500U, e 4GB di RAM DDR3.
Il portatile utilizzato si occupava, oltre che di gestire l'algoritmo di controllo, di visualizzare output di debug.
Per il controllo del drone è stato utilizzato il driver \textit{ardrone\_autonomy}, disponibile nei repository di ROS.
Per muovere il quadrotor è stato utilizzato un joystic, integrato con ROS tramite il nodo \textit{joy}, anchesso presente nei repository ufficiali di ROS.
Per impartire comandi dal joystick al quadrotor ci si è avvalsi di un controllore già implementato nel package  \textit{ardrone\_tutorials}, che fornisce anche una finestra che mostra l'immagine presa direttamente dalla telecamera del drone.

\section{Classificatore utilizzato}

%%TODO classificatore

