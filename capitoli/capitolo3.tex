\chapter{Concetti Fondamentali}
\label{cap:concetti}
\thispagestyle{empty}

\begin{quotation}
{\footnotesize
\noindent{\emph{Doc: Ovviamente il continuum temporale è stato interrotto creando questa nuova temporale sequenza di eventi risultanti in questa realtà alternativa. \\
Marty: Che lingua è Doc? \\
Doc: Ah si, si, si, si, si! Lascia che ti spieghi. Immagina, che questa linea rappresenti il tempo\dots}}
\begin{flushright}
Ritorno al Futuro, parte II
\end{flushright}
}
\end{quotation}
\vspace{0.5cm}

Si illustrano di seguito i concetti fondamentali alla comprensione della tesi.

\section{Logica Fuzzy}
La logica fuzzy è una logica multivalore che permette di esprimere non solo la verità o la falsità di una affermazione, ma anche il suo valore di verità, che può essere qualsiasi numero tra 0 (ossia l'afefrmazione è falsa) e 1 (ossia l'affermazione è vera).

Grazie a questa logica è possibile esprimere concetti che in una logica a due valori sono paradossali.
Ad esempio, la frase ``sono un bugiardo'', che nella logica a due valori risulta paradossale, nella logica fuzzy essa può assumere valori consistenti.

\subsection{Insiemi Fuzzy}
La logica fuzzy è basata fortemente sul concetto di insieme fuzzy. Un insieme fuzzy è un insieme la cui funzione di appartenenza può avere valori nell'intervallo $[0, 1]$.
I fuzzy set sono quindi definiti da una funzione di appartenenza $f$ definita su un dominio $D$, avente come codominio l'intervallo $[0,  1]$.  Formalmente un fuzzy et è definito come:

\begin{equation*}
 f:D\leftarrow [0, 1]
\end{equation*}

Si definisce altezza di un insieme fuzzy il massimo grado di verità possibile per un elemento.
Un fuzzy set si dice normale se la sua altezza è pari a uno.

Un fuzzy set si dice unimodale se esiste un unico insieme in cui le variabili hanno valore di verità pari a 1.


%%TODO qui?? forse da spostare...
Si definisce variabile linguistica $X$, definita sulla variabile di base $u$ una tupla così definita:
\begin{equation*}
$(X, T(X), U, G, M)$
\end{equation*}

dove:

\begin{description}
 \item [X] è il nome della variabile.
 \item [T(X)] è un insieme di termini, detti anche valori linguistici.
 \item [U] è il dominio della variabile di base $u$.
 \item [G] è una regola sintattica per generare l'interpretazione X per ogni valore di u.
 \item [M] è una regola semantica che associa a X il suo significato.
\end{description}


Data una variabile $x$, è possibile definire più fuzzy set rispetto al suo dominio. Si parla di frame of cognition quando valgono le seguenti condizioni:
\begin{enumerate}
 \item L'intero dominio della variabile è coperto da almeno un fuzzy set con valore di verità maggiore di zero.
 \item Tutti i fuzzy set sono unimodali.
 \item Tutti i fuzzy set sono normali.
\end{enumerate}

Inoltre si parla di partizione fuzzy, se la somma delle funzioni di appartenenza di ogni fuzzy set in ogni punto del dominio è uguale a 1.

La normale teoria degli insiemi è contenuta nella teoria degli insiemi fuzzy: Infatti un intervallo è un insieme fuzzy, con valore di verità di tutti gli elementi pari a 1. Anche i singoli elementi possono essere trattati, tramite il fuzzy set singleton, che ha valore di verità 1 per un solo elemento del dominio, e 0 per tutti gli altri.

\subsection{Operazioni sui fuzzy set}

Sui fuzzy set sono definite le seguenti operazioni:
\begin{itemize}
 \item unione
 \item intersezione
 \item complemento
\end{itemize}

\subsubsection{Complemento}

Dato un insieme fuzzy $A$, con funzione di appartenenza $\mu_A$ il complemento è definito come:

\begin{equation*}
 c(\mu_A(x)) = \mu_{\neg A}(x)
\end{equation*}

Il complemento di insiemi fuzzy deve rispettare i seguenti assiomi:
\begin{enumerate}
 \item $c(0)=1$, $c(1)=0$ (condizioni al contorno)
 \item $c$ è una funzione continua.
 \item $c$ è involutiva, ossia $c(c(a))=a\, \, \forall a \in [0, 1]$
\end{enumerate}

l'unica funzione che rispetta i seguenti assiomi è la funzione:

\begin{equation*}
 c(x) = 1 - x
\end{equation*}

\subsubsection{Intersezione}

Dati due insiemi fuzzy $A$ e $B$, con funzione di appartenenza $\mu_A$ e $\mu_B$ l'intersezione  è definita come:

\begin{equation*}
 i(\mu_A(x), \mu_B(x)) = \mu_{A\cap B}(x)
\end{equation*}

L'intersezione di insiemi fuzzy deve rispettare i seguenti assiomi:

\begin{enumerate}
 \item $i(a, 1)=a$ (condizioni al contorno)
 \item $d\geq b \implies  i(a,d)\geq i(a,b)$ (monotonicità)
 \item $i(b,a) = i(a,b)$ (commutatività)
 \item $i(i(a,b),d)=i(a,i(b,d))$ (associatività)
 \item $i$ è una funzione continua
 \item $a\geq i(a,a)$ (sub-idempotenza)
 \item $a_1<a_2 \wedge b_1<b_2\implies i(a_1,b_1)<i(a_2,b_2)$ (monotonicità stretta)
\end{enumerate}

Non esiste un'unica funzione che rispetta questi assiomi. Ogni funzione che rispetta questi assiomi viene detta T-norm.
Esempi di T-norm sono l'operatore minimo e il prdotto tra i valori di verità.

\subsubsection{Unione}

Dati due insiemi fuzzy $A$ e $B$, con funzione di appartenenza $\mu_A$ e $\mu_B$ l'unione è definita come:

\begin{equation*}
 u(\mu_A(x), \mu_B(x)) = \mu_{A\cup B}(x)
\end{equation*}

L'unione di insiemi fuzzy deve rispettare i seguenti assiomi:

\begin{enumerate}
 \item $u(a, 0)=a$ (condizioni al contorno)
 \item $b \leq d implies u(a,b) \leq u(a,d)$ (monotonicità)
 \item $u(a,b) = u(b,a)$ (commutatività)
 \item $u(a,u(b,d)) = u(u(a,b),d)$ (associatività)
 \item $u$ è una funzione continua
 \item $u(a,a) \geq a$ (super-idempotenza)
 \item $a_1< a_2 \wedge b_1 < b_2 \implies u(a_1,b_1)<u(a_2,b_2)$ (monotonicità stretta)
\end{enumerate}

Non esiste un unica funzione che rispetta questi assiomi. Ogni funzione che rispetta questi assiomi viene detta T-conorm, o S-norm.
Esempi di S-norm sono l'operatore massimo e la somma probabilistica tra i valori di verità.
La somma probabilistica $p$ tra due variabili $a$  e $b$ è definita come:

\begin{equation*}
 p = a+b - a\cdot b 
\end{equation*}

%%TODO aggregazione???

% μA(x) = h[μA1(x), ..., μ An(x)]
% Axioms:
% 1. h[0,..., 0]=0, h[1,..., 1]=1 (boundary conditions)
% 2. monotonicity
% 3. h is continuous
% 4. h(a,..,a) = a (idempotency)
% 5. simmetricity
% Example of aggregation operator: generalized average
% h(a1, ..., an) = (a1α + ...+ anα) 1/α / n


\subsection{Regole Fuzzy}

Una regola di inferenza fuzzy può essere considerata come un modello, un mezzo per definire una funzione tra un ingresso e una uscita.

Le regole fuzzy che consideriamo sono composte nel seguente modo:
\begin{verbatim}
 IF <antecedente> THEN <consequente>
\end{verbatim}

Dove l'antecedente è una formula logica ben formata è il conseguente è una o più proposizioni legate dall'operatore and.

Una formula ben formata è definita come:

\begin{itemize}
 \item Una proposizione è una formula ben formata
 \item Se ``A'' è una formula ben formata, anche ``not A'' è una formula ben formata.
 \item Se ``A'' e ``B'' sono formule ben formate anche ``A and B'', ``A or B'' sono formule ben formate.
 \item null'altro è una formula ben formata.
\end{itemize}

Una proposizione è definita come 
