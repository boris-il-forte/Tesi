\chapter{Concetti Fondamentali}
\label{cap:concetti}
\thispagestyle{empty}

\begin{quotation}
{\footnotesize
\noindent{\emph{Doc: Ovviamente il continuum temporale è stato interrotto creando questa nuova temporale sequenza di eventi risultanti in questa realtà alternativa. \\
Marty: Che lingua è Doc? \\
Doc: Ah si, si, si, si, si! Lascia che ti spieghi. Immagina, che questa linea rappresenti il tempo\dots}}
\begin{flushright}
Ritorno al Futuro, parte II
\end{flushright}
}
\end{quotation}
\vspace{0.5cm}

Si illustrano di seguito i concetti fondamentali alla comprensione della tesi.

\section{Logica Fuzzy}
La logica fuzzy è una logica multivalore che permette di esprimere non solo la verità o la falsità di una affermazione, ma anche il suo valore di verità, che può essere qualsiasi numero tra 0 (ossia l'afefrmazione è falsa) e 1 (ossia l'affermazione è vera).

Grazie a questa logica è possibile esprimere concetti che in una logica a due valori sono paradossali.
Ad esempio, la frase ``sono un bugiardo'', che nella logica a due valori risulta paradossale, nella logica fuzzy essa può assumere valori consistenti.

\subsection{Insiemi Fuzzy}
La logica fuzzy è basata fortemente sul concetto di insieme fuzzy. Un insieme fuzzy è un insieme la cui funzione di appartenenza può avere valori nell'intervallo $[0, 1]$.
I fuzzy set sono quindi definiti da una funzione di appartenenza $f$ definita su un dominio $D$, avente come codominio l'intervallo $[0,  1]$.  Formalmente un fuzzy et è definito come:

\begin{equation*}
 f:D\leftarrow [0, 1]
\end{equation*}

Si definisce altezza di un insieme fuzzy il massimo grado di verità possibile per un elemento.
Un fuzzy set si dice normale se la sua altezza è pari a uno.

Un fuzzy set si dice unimodale se esiste un unico insieme in cui le variabili hanno valore di verità pari a 1.


%%TODO qui?? forse da spostare...
Si definisce variabile linguistica $X$, definita sulla variabile di base $u$ una tupla così definita:
\begin{equation*}
(X, T(X), U, G, M)
\end{equation*}

dove:

\begin{description}
 \item [X] è il nome della variabile.
 \item [T(X)] è un insieme di termini, detti anche valori linguistici.
 \item [U] è il dominio della variabile di base $u$.
 \item [G] è una regola sintattica per generare l'interpretazione X per ogni valore di u.
 \item [M] è una regola semantica che associa a X il suo significato.
\end{description}


Data una variabile $x$, è possibile definire più fuzzy set rispetto al suo dominio. Si parla di frame of cognition quando valgono le seguenti condizioni:
\begin{enumerate}
 \item L'intero dominio della variabile è coperto da almeno un fuzzy set con valore di verità maggiore di zero.
 \item Tutti i fuzzy set sono unimodali.
 \item Tutti i fuzzy set sono normali.
\end{enumerate}

Inoltre si parla di partizione fuzzy, se la somma delle funzioni di appartenenza di ogni fuzzy set in ogni punto del dominio è uguale a 1.

La normale teoria degli insiemi è contenuta nella teoria degli insiemi fuzzy: Infatti un intervallo è un insieme fuzzy, con valore di verità di tutti gli elementi pari a 1. Anche i singoli elementi possono essere trattati, tramite il fuzzy set singleton, che ha valore di verità 1 per un solo elemento del dominio, e 0 per tutti gli altri.

\subsection{Operazioni sui fuzzy set}

Sui fuzzy set sono definite le seguenti operazioni:
\begin{itemize}
 \item unione
 \item intersezione
 \item complemento
 \item aggregazione
\end{itemize}

\subsubsection{Complemento}

Dato un insieme fuzzy $A$, con funzione di appartenenza $\mu_A$ il complemento è definito come:

\begin{equation*}
 c(\mu_A(x)) = \mu_{\neg A}(x)
\end{equation*}

Il complemento di insiemi fuzzy deve rispettare i seguenti assiomi:
\begin{enumerate}
 \item $c(0)=1$, $c(1)=0$ (condizioni al contorno)
 \item $c$ è una funzione continua.
 \item $c$ è involutiva, ossia $c(c(a))=a\, \, \forall a \in [0, 1]$
\end{enumerate}

l'unica funzione che rispetta i seguenti assiomi è la funzione:

\begin{equation*}
 c(x) = 1 - x
\end{equation*}

\subsubsection{Intersezione}

Dati due insiemi fuzzy $A$ e $B$, con funzione di appartenenza $\mu_A$ e $\mu_B$ l'intersezione  è definita come:

\begin{equation*}
 i(\mu_A(x), \mu_B(x)) = \mu_{A\cap B}(x)
\end{equation*}

L'intersezione di insiemi fuzzy deve rispettare i seguenti assiomi:

\begin{enumerate}
 \item $i(a, 1)=a$ (condizioni al contorno)
 \item $d\geq b \implies  i(a,d)\geq i(a,b)$ (monotonicità)
 \item $i(b,a) = i(a,b)$ (commutatività)
 \item $i(i(a,b),d)=i(a,i(b,d))$ (associatività)
 \item $i$ è una funzione continua
 \item $a\geq i(a,a)$ (sub-idempotenza)
 \item $a_1<a_2 \wedge b_1<b_2\implies i(a_1,b_1)<i(a_2,b_2)$ (monotonicità stretta)
\end{enumerate}

Non esiste un'unica funzione che rispetta questi assiomi. Ogni funzione che rispetta questi assiomi viene detta T-norm.
Esempi di T-norm sono l'operatore minimo e il prdotto tra i valori di verità.

\subsubsection{Unione}

Dati due insiemi fuzzy $A$ e $B$, con funzione di appartenenza $\mu_A$ e $\mu_B$ l'unione è definita come:

\begin{equation*}
 u(\mu_A(x), \mu_B(x)) = \mu_{A\cup B}(x)
\end{equation*}

L'unione di insiemi fuzzy deve rispettare i seguenti assiomi:

\begin{enumerate}
 \item $u(a, 0)=a$ (condizioni al contorno)
 \item $b \leq d \implies u(a,b) \leq u(a,d)$ (monotonicità)
 \item $u(a,b) = u(b,a)$ (commutatività)
 \item $u(a,u(b,d)) = u(u(a,b),d)$ (associatività)
 \item $u$ è una funzione continua
 \item $u(a,a) \geq a$ (super-idempotenza)
 \item $a_1< a_2 \wedge b_1 < b_2 \implies u(a_1,b_1)<u(a_2,b_2)$ (monotonicità stretta)
\end{enumerate}

Non esiste un unica funzione che rispetta questi assiomi. Ogni funzione che rispetta questi assiomi viene detta T-conorm, o S-norm.
Esempi di S-norm sono l'operatore massimo e la somma probabilistica tra i valori di verità.
La somma probabilistica $p$ tra due variabili $a$  e $b$ è definita come:

\begin{equation*}
 p = a+b - a\cdot b 
\end{equation*}

\subsection{Aggregazione}

Dato un insieme di fuzzy set $(A_1, \dots, A_n)$, con funzione di appartenenza $(\mu_{A_1}, \dots, \mu_{A_n})$ l'insieme fuzzy aggregato $A$ è definito come:

\begin{equation*}
   \mu_A(x) = h(\mu_{A_1}(x), \dots, \mu_{A_n}(x))
\end{equation*}

L'aggregazione di fuzzy set deve rispettare i seguenti assiomi:

\begin{enumerate}
 \item $h(0, \dots, 0)=0, h(1, \dots, 1)=1 $ (condizioni al contorno)
 \item $a_i \leq b_i\, \forall i \implies h(a_1, a_n) \leq h(b_1, b_n)$ (monotonicità)
 \item $h$ è una funzione continua
 \item $h(a, \dots,a)=a$ (idempotenza)
 \item $h(a_1,\dots,a_n) = h(a_i, \dots, a_j)\, \forall i,\dots, j\in [1,n]$ (simmetria) %%
\end{enumerate}

Non esiste un unica funzione che rispetta questi assiomi. Un esempio di operatore di aggregazione valido è la media generalizzata:
\begin{equation*}
 h(a_1, \dots, a_n) = \dfrac{(a_1^\alpha + \dots + a_n^\alpha)^{1/\alpha}}{n}
\end{equation*}



\subsection{Regole Fuzzy}

Una regola di inferenza fuzzy può essere considerata come un modello, un mezzo per definire una funzione tra un ingresso e una uscita.

Le regole fuzzy che consideriamo sono composte nel seguente modo:
\begin{verbatim}
 IF <antecedente> THEN <consequente>
\end{verbatim}

Dove l'antecedente è una formula logica ben formata è il conseguente è una o più proposizioni legate dall'operatore and.

Una formula ben formata è definita come:

\begin{itemize}
 \item Una proposizione è una formula ben formata
 \item Se ``A'' è una formula ben formata, anche ``not A'' è una formula ben formata.
 \item Se ``A'' e ``B'' sono formule ben formate anche ``A and B'', ``A or B'' sono formule ben formate.
 \item null'altro è una formula ben formata.
\end{itemize}

Una proposizione è definita come un assegnamento di una etichetta a una variabile fuzzy, del tipo
\begin{verbatim}
 <VARIABLE> IS <LABEL>
\end{verbatim}

esistono due tipi di regole:
\begin{description}
 \item [Regole linguistiche] note anche come regole di Mamdami, dove il conseguente è la congiunzione di più clausole linguistiche. Queste regole possono essere considerate come una funzione che mappa una configurazione di input in ingresso a una interpretazione simbolica dell'output desiderato. 
 \item [Regole di modello] note anche come regole di Takagi-Sugeno-Kosko. In questo tipo di regole le proposizioni nel conseguente sono del tipo: \verb|<VARIABLE> IS f(...)|. Questo tipo di regole legano un modello alla variabile di uscita. la funzione \verb|f| può essere definita in qualunque modo, e prende in input le variabili di base delle variabili linguistiche pesenti sul lato sinistro della regola.

\end{description}


Un insieme di regole fuzzy forma una knowledgebase. Le regole della knowledgebase possono avere diversi pesi, in modo da rendere più importanti alcune regole specifiche, oppure evitare la preponderanza delle regole lunghe rispetto a quelle corte.
L'inferenza sulla knowledgebase viene fatta come segue: 

\begin{enumerate}
 \item Dato un insieme di fatti noti, ossia variabili linguisiche di cui si conosce il valore, vengono selezionate solo le regole il cui antecedente contiene solo le variabili note.
 \item Viene calcolato il valore di verità del lato sinistro delle regole
 \item Se specificato, il valore di verità viene combinato con il peso della regola.
 \item Viene aggregato l'output da regole diverse che calcolano la stessa etichetta.
 \item Viene defuzzyficato, eventualmente, l'output simbolico calcolato, oppure viene compiuta una approssimazione linguistica, ossia viene calcolata l'etichetta in output che meglio approssima, decisa una metrica, l'output simbolico calcolato (che può contenere più etichette per la stessa variabile)
\end{enumerate}

Esistono vari tipi di defuzzificazione. L'operazione di defuzzificazione viene applicata all'insieme di insiemi fuzzy ricavati per ciascuna variabile linguistica. I metodi più noti sono:
\begin{itemize}
 \item Centro di massa
 \item Bisettrice
 \item Media dei massimi
 \item Minor massimo
 \item Maggior massimo
 \item Centro dell'area più alta
\end{itemize}

Il metodo più usato in assoluto è il centro di massa applicato a singleton.

\section{Geometria della telecamera}

Verrà descritta brevemente la geometria della telecamera.
Una telecamera è un dispositivo in grado di acquisire immagini del mondo. Essa è composta da una lente e un piano dell'immagine, in grado di acquisire l'immagine. Nel nostro modello consideriamo la lente come estremamente sottile, in modo da non considerare le distorsioni radiali e tangenziali dovute alla lente. La lente è necessara in quanto essa è in grado di convogliare tutta la luce proveniente da un determinato punto dello spazio in un'unico punto (oppure in un'area di punti) nel piano immagine.

Ciamiamo asse ottico l'asse perpendicolare al piano dell'immagine passante per il centro della lente. Si chiama Principal Point l'intersezione tra l'asse ottico e il piano dell'immagine.
Chiamiamo distanza focale la distanza tra la lente e il piano dell'immagine.

\subsection{Geometria proiettiva}

La relazione tra le coordinate della telecamera e le coordinate dei punti del mondo è una relazione non lineare.
Dato un punto nello spazio $P=(X,Y,Z)$, la sua proiezione nel piano dell'immagine $p=(x,y)$ avrà le seguenti coordinate:

\begin{equation*}
 \begin{aligned}
  x = f\cdot \dfrac{X}{Z} \\
  y = f\cdot \dfrac{Y}{Z}
 \end{aligned}
\end{equation*}

Per semplificare il problema, viene usato un'altro sistema di coordinate in cui la relazione considerata è una relazione lineare.
Queste coordinate sono dette coordinate omogenee, e uno spazio che le utilizza viene detto spazio proiettivo.

Le coordinate omogenee sono delle coordinate che per esprimere uno spazio a $n$ dimensioni, utilizzano $n+1$ coordinate. Le coordinate spaziali sono definite a meno di un fattore moltiplicativo, ovvero $P = \alpha\cdot P$.
La quarta coordinata ha inoltre un significato particolare. Se è diversa zero il punto rappresenta un punto reale dello spazio euclideo, altrimenti rappresenta una direzione, detto anche punto all'infinito.


\subsubsection{Piano immagine}
Modelliamo l'immagine con uno spazio proiettivo a due dimensioni, $\mathcal{P}^2$.
Un punto $X$ del piano immagine è definito come:
\begin{equation*}
  X=\colvec{3}{x}{y}{w}
\end{equation*}

E' possibile anche definire una rettta $l$ nel piano immagine sempre utilizzando il vettore tridimensionale:

\begin{equation*}
  l=\colvec{3}{a}{b}{c}
\end{equation*}

Un punto $X$ appartiene a una retta $l$ se:

\begin{equation*}
 X^T\cdot l = 0
\end{equation*}

Equivalentemente si può dire che una retta $l$ passa per un punto $X$ se:

\begin{equation*}
 l^T\cdot X = 0
\end{equation*}

l'equazione di una retta è quindi definita come:

\begin{equation*}
 a\cdot x + b \cdot y+ c \cdot w = 0
\end{equation*}

Anche la retta quindi è definita a meno di un coefficiente moltiplicativo.
La retta che contiene tutti i punti all'infinito è della linea all'infinito $l_\infty$.

E' possibile esprimere anche coniche nel piano proiettivo. Una conica $C$ è definita come:

\begin{equation*}
 C = \begin{pmatrix} a & b & c \\ b & d & e \\ c & e & f \\ \end{pmatrix}
\end{equation*}

Si può notare che la matrice è simmetrica. Questo avviene senza perdita di generalità.
Un punto appartiene a una conica se:

\begin{equation*}
 X^T\cdot C \cdot X = 0
\end{equation*}

Anche le coniche sono definite a meno di un coefficiente moltiplicativo.

Infine è possibile esprimere coniche duali. Se le conichesono insieme di punti, le coniche duali sono insieme di linee.
Si può dimostrare che la conica duale $C^*$ alla conica $C$ è:
\begin{equation*}
 C^*=C^{-1}
\end{equation*}

una retta appartiene alla conica duale se:

\begin{equation*}
 l^T\cdot C^* \cdot l = 0
\end{equation*}

Lo spazio proiettivo a 2 dimensioni ha due punti notevoli detti punti circolari. Tutte le possibili circonferenze si intersecano nei punti circolari, che sono due punti immaginari:

\begin{equation*}
 \begin{aligned}
  I=\colvec{3}{1}{i}{0} & & J=\colvec{3}{1}{-i}{0}
 \end{aligned}
\end{equation*}

I punti circolari appartengono a $l_\infty$.

Questi due punti possono essere espressi in maniera compatta dalla conica duale ai punti cicolari, una conica degenere, che può essere calcolata come:

\begin{equation*}
 C^*_\infty = I\cdot J^T + J\cdot I^T =\begin{pmatrix} 1 & 0 & 0 \\ 0 & 1 & 0 \\ 0 & 0 & 0 \\ \end{pmatrix}
\end{equation*}


\subsubsection{Mondo reale}
Modelliamo il mondo reale con uno spazio proiettivo a 3 dimensioni, $\mathcal{P}^3$.

Un punto $X$ del mondo reale è definito come:
\begin{equation*}
  X=\colvec{4}{x}{y}{z}{w}
\end{equation*}

E' possibile anche definire piano $\pi$ nel mondo reale utilizzando il vettore quadridimensionale:

\begin{equation*}
  \pi=\colvec{4}{a}{b}{c}{d}
\end{equation*}

Un punto $X$ appartiene a una piano $\pi$ se:

\begin{equation*}
 X^T\cdot \pi = 0
\end{equation*}

Equivalentemente si può dire che un piano $\pi$ passa per un punto $X$ se:

\begin{equation*}
 \pi^T\cdot X = 0
\end{equation*}

l'equazione di un piano è quindi definita come:

\begin{equation*}
 a\cdot x + b \cdot y+ c \cdot z + d \cdot w = 0
\end{equation*}

Anche il piano quindi è definito a meno di un coefficiente moltiplicativo.
Il piano che contiene tutti i punti all'infinito e, conseguentemente, tutti le rette all'infinito di tutti i piani dello spazio, si chiama piano all'infinito $\pi_\infty$. 

E' possibile esprimere anche quadriche nel piano proiettivo. Una quadrica $Q$ è definita come:

\begin{equation*}
 Q = \begin{pmatrix} a & b & c & d \\ b & e & f & g \\ c & f & h & i \\ d & g & i & l \\ \end{pmatrix}
\end{equation*}

Si può notare che la matrice è simmetrica. Questo avviene senza perdita di generalità.
Un punto appartiene a una quadrica se:

\begin{equation*}
 X^T\cdot Q \cdot X = 0
\end{equation*}

Anche le quadriche sono definite a meno di un coefficiente moltiplicativo.

Infine è possibile esprimere quadriche duali. Se le quadriche sono insieme di punti, le quadriche duali sono insieme di piani.
Si può dimostrare che la quadrica duale $Q^*$ alla quadrica $Q$ è:
\begin{equation*}
 Q^*=Q^{-1}
\end{equation*}

un piano appartiene alla quadrica duale se:

\begin{equation*}
 \pi^T\cdot C^* \cdot \pi = 0
\end{equation*}

Lo spazio proiettivo a 3 dimensioni ha una conica notevole, detta conica assoluta, $\Omega$. La conica assoluta è l'intersezione di tutte le possibili sfere nel mondo reale. La conica assoluta è formata dall'unione di tutti i possibili punti circolari calcolati lungo tutti i possibili piani dello spazio.
La conica assoluta è contenuta in $\pi_\infty$.

La conica assoluta può essere espressa in maniera sintetica con la quadrica duale assoluta $\Omega^*$:

\begin{equation*}
 \Omega^*_\infty = I\cdot J^T + J\cdot I^T =\begin{pmatrix} 1 & 0 & 0 & 0 \\ 0 & 1 & 0 & 0 \\ 0 & 0 & 1 & 0 \\ 0 & 0 & 0 & 0 \\ \end{pmatrix}
\end{equation*}

\subsection{Trasformazioni Proiettive}

Si chiama trasformazione proiettiva una qualsiasi trasformazione lineare di coordinate di uno spazio proiettivo.
Generalmente una trasformazione proiettiva è una trasformazione non lineare rispetto allo spazio euclideo che rappresenta tutti i punti che non sono all'infinito.
Una trasformazione proiettiva di coordinate è data dalla formula:
\begin{equation*}
 x' = \mathcal{H}(x) = H\cdot x
\end{equation*}

Nello spazio $\mathcal{P}^2$ le trasformazioni proiettive sono descritte da matrici $H$ $3\times3$, mentre nello spazio $\mathcal{P}^3$ le trasformazioni proiettive sono descritte da matrici $H$ $4\times4$.

Ci sono alcuni sottocasi di trasformazione propettica importanti e utili anche ai fini di questa tesi:

\begin{itemize}
 \item Rotazioni
 \item Traslazioni
 \item Isometrie
 \item Similitudini
 \item Affinità
\end{itemize}

verranno discusse in seguito con i loro invarianti.

\subsubsection{Isometrie}

Le isometrie sono le trasformazioni prospettiche più semplici. Esse sono sostanzialmente rototraslazioni, e quindi contengono come casi particolari le rotazioni e le traslazioni pure. Gli invarianti delle isometrie sono le lunghezze, gli angoli, la forma e la dimensione, la posizione relativa tra gli oggetti. Hanno 3 gradi di libertà nel piano e 6 nello spazio.

una similitudine è descritta dalla matrice:

%%TODO matrice 

\subsubsection{Similitudini}
Le similitudini sono trasformazioni prospettiche più generali delle isometrie. Hanno un grado di libertà in più rispetto alle isometrie, essendo la scala degli oggetti non fissata. Gli invarianti sono il rapporto tra le lnghezze, gli angoli, la forma degli oggetti.

%%TODO matrice 

\subsubsection{Affinità}
Le affinità sono ancora piùà generali rispetto alle similitudini. Hanno 6 gradi di libertà nel piano e 9 nello spazio. Gli invarianti della trasformazione sono l parallelismo e il rapporto tra due segmenti paralleli.

%%TODO matrice 

\subsection{Geometria della telecamera e calibrazione}

La telecamera è uno strumento che trasforma dei punti nello spazio a 3 dimensioni in punti sull'immagine a 2 dimensioni. La telecamera dunque attua una trasformazione proiettiva $P$ tra i due spazi.
L'equzione che mappa i punti tra i due spazi è la seguente:
\begin{equation*}
 \colvec{3}{x}{y}{w} = P\cdot\colvec{4}{X}{Y}{Z}{W}
\end{equation*}

la matrice $P$ dipende da due fattori: i parametri intrinseci della telecamera e i parametri estrinseci.
I primi dipendono dalla costruzione della fotocamera, e sono la distanza focale, la dimensione e la forma dei pixel, la posizione del principal point. I secondi sono la posa della telecamera rispetto al mondo, ovvero la rototraslazione della telecamera rispetto al mondo.  
Possiamo allora scrivere $P$ nel seguente modo:
\begin{equation*}
 P = K \cdot \left[
    \begin{array}{c|c}
      R & t
    \end{array} 
\right]
\end{equation*}

Dove la matrice $K$ è la matrice dei parametri intrinseci della telecamera, detta anche matrice di calibrazione, $R$ e $t$ sono, rispettivamente, la matrice di rotazione ($3\times3$) e traslazione ($3\times1$) della camera rispetto alle coordinate  
La calibrazione della camera, e quindi il calcolo di $K$ si basa sulle proprietà dell'immagine della conica assoluta, $\omega$.
L'immagine della conica assoluta è 

