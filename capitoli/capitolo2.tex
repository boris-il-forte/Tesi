\chapter{Stato dell'arte}
\label{capitolo2}
\thispagestyle{empty}

\begin{quotation}
{\footnotesize
\noindent{\emph{``Terence: Tu lo reggi il whisky? \\
Bud: Beh, i primi due galloni si, al terzo divento nostalgico e ci pu\`o scappare la lite... E tu lo reggi? \\
Terence: Eh, che domande, io sono stato allattato a whisky!''
} }
\begin{flushright}
I due superpiedi quasi piatti
\end{flushright}
}
\end{quotation}
\vspace{0.5cm}

Il problema della localizzazione di un robot in un ambiete sconosciuto è stato affrontato fin dagli anni 90' a partire da \cite{174711}, articolo nel quale per la prima volta si delineava un framework per localizzare un robot costruendo contemporaneamente la mappa dell'ambiente.Tramite l'utilizzo di sonar, venivano estratte feature geometriche con cui veniva costruita la mappa, nella quale il robot si localizzava. Il problema principale della localizzazione è il ``problema della correlazione'': se la posizione della feature rispetto alla quale ci si localizza è affetta da incertezza, la conseguente stima della posizione effettuata rispetto a tale feature sarà affetta da un errore che dipende dall'errore della posizione della feature %stessa. Questo problema diventa tanto più grave se si pensa che la posizione del robot in ogni istante non è nota a priori, ma deve essere stimata sulla base delle osservazioni precedenti. Come è facile vedere, è necessario risolvere questo problema per evitare che l'errore 
della generazione della mappa e l'errore della stima della posizione divergano nel tempo. Per risolvere questo problema, gli autori hanno utilizzato un filtro di Kalman esteso.
Come è noto, il filtro di Kalman è uno stimatore Bayesiano ricorsivo, che, supposto noto il modello lineare che regola la generazione dei dati e la loro osservazione, supposto che l'errore di misura e di modello siano gaussiani, restituisce la densità di probabilità del sistema osservato. Il filtro di Kalman, se utilizzato secondo le ipotesi, è uno stimatore ottimo dello stato del sistema osservato, secondo i minimi quadrati.
Tuttavia, nell'ambito della robotica, e in particolare nel problema della localizzazione, il modello di %generazione e osservazione dei dati non può essere considerato lineare. E' quindi necessario utilizzare un'estensione del filtro di kalman al caso non lineare: il filtro di Kalman esteso (EKF) è una delle possibili soluzioni al problema. L'idea alla base del filtro di Kalmen esteso è quella di lavorare sul modello linearizzato, stimato ricorsivamente dal modello non lineare sulla base della stima corrente.
